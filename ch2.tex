\chapter{Computing the Sum and Mean of Vectors in C++ and R}

\section{Sum}
Compute the sum of a vector.  The sum1 function uses a for loop and the sum2 function uses std::accumulate from the STL.

\lstset{language=C++}
\lstinputlisting{code/vec_sum.cpp}

\section{Cumulative Sum}
Compute the cumulative sum of a vector. The cumsum1 function uses a for loop whereas the cumsum2 function uses std::partial\_sum from the STL.

\lstset{language=C++}
\lstinputlisting{code/vec_cumsum.cpp}

\section{Mean}
Compute the mean of a vector. Note how I use the same sum2 function that was defined earlier by copy/pasting it into the file. In general, this is a bad thing to do, but I am doing this for simplicity and so each file can be run as a stand-alone program. The correct way would be to define the sum2 function in it's own .cpp file, then declare it in a header file, then include the header file.

\lstset{language=C++}
\lstinputlisting{code/vec_mean.cpp}